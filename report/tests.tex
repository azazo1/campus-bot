在本项目中,我们采用了 Python 标准库中的 \texttt{unittest} 模块进行单元测试,以确保项目核心功能的稳定性和正确性。

\begin{itemize}
    \item \textbf{测试环境的隔离}:通过 \texttt{setUp()} 和 \texttt{tearDown()} 方法,\texttt{unittest} 提供了对每个测试用例的独立初始化和资源回收,避免测试用例之间的相互干扰。
    \item \textbf{断言机制}:\texttt{unittest} 提供了多种断言方法(如 \texttt{assertEqual()}、\texttt{assertTrue()}),用以验证测试结果是否符合预期。
\end{itemize}

\textbf{测试用例的结构设计}:

以下是测试模块的代码结构:
\begin{lstlisting}[language=Python]
class TestCalendar(unittest.TestCase):
    def setUp(self):
        init()  # 初始化日志记录器
        self.cache = load_cache()  # 加载登录缓存
        self.calendar = CalendarQuery(self.cache.get_cache(PortalCache))

    def test_user_schedules(self):
        now = datetime.datetime.now()
        pprint(self.calendar.query_user_schedules(
            int(now.timestamp() * 1000),
            int((now + datetime.timedelta(days=1)).timestamp() * 1000),
        ))

    def test_school_calendar(self):
        school_calendar = self.calendar.query_school_calendar()
        pprint(school_calendar)
\end{lstlisting}

\textbf{测试框架的执行流程}:
\begin{itemize}
    \item 在 \texttt{setUp()} 方法中完成初始化工作,包括日志系统的初始化和登录缓存的加载。
    \item 使用 \texttt{test\_user\_schedules()} 方法测试用户日程的获取功能,通过时间戳计算和校验确保数据范围的准确性。
    \item 使用 \texttt{test\_school\_calendar()} 方法验证校历的查询功能,确保校历数据的完整性和准确性。
\end{itemize}
